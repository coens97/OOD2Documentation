\chapter{Functional Requirements}
\section{Requirements}
\subsection{General requirements}
\begin{tabularx}{\textwidth}{|p{2cm}X|}\hline
Code & Requirement \\\hline
GEN-010 & The program is compatible with Windows 7.\\\hline
GEN-020 & The system allows to design a flow system using the folowing components:
\begin{itemize}[noitemsep,nolistsep]
\item Pump
\item Sink
\item Adjustable splitter 
\item Merger
\end{itemize}
\\\hline
GEN-030 & The system shows the flow through the pipelines.\\\hline
\end{tabularx}

\subsection{System requirements}
\begin{tabularx}{\textwidth}{|p{2cm}X|}\hline
Code & Requirement \\\hline
SYS-010 & The system allows to add components to the flow network.\\\hline
SYS-015 & The system allows to duplicate components (excluding pipelines).\\\hline
SYS-018 & The system allows to move components.\\\hline
SYS-020 & The components are not allowed to overlap each other.\\\hline
SYS-030 & The system allows to connect a pipeline between an unused output of a component to an unused input of a component.\\\hline
SYS-040 & The system allows to remove components including pipelines.\\\hline
SYS-040A & All the pipelines connected to the removed component will also be removed.\\\hline
SYS-050 & The system allows to change the flow output of a pump within his range.\\\hline
SYS-060 & The system allows to change the distribution of the flow from splitter going to the top and bottom output.\\\hline
SYS-070 & The system displays the flow going through the pipelines.\\\hline
SYS-070A & The system displays the pipeline red in case the flow exceeds the safety limit.\\\hline
SYS-080 & The system displays the flow transported to a sink.\\\hline
SYS-090 & The system allows to export the flow network to a file.\\\hline
SYS-100 & The system allows to import a flow network from a file.\\\hline
\end{tabularx}

\section{Use cases}

\begin{usecase}

\addtitle{Use Case 1}{Adding new component on flow network} 

%Level: "user-goal" or "subfunction"
\addfield{Level:}{User-goal}

%Primary Actor: Calls on the system to deliver its services.
\addfield{Primary Actor:}{End-User}

%Preconditions: What must be true on start and worth telling the reader?
\addfield{Preconditions:}{Component selected from toolbar}

%Main Success Scenario: A typical, unconditional happy path scenario of success.
\addscenario{Main Success Scenario:}{
	\item User selects place on flow network grid
	\item System places component on grid with default values see section \ref{sec:defv}
	\item System updates internal state
}

%Extensions: Alternate scenarios of success or failure.
\addscenario{Extensions:}{
	\item[2.a] System can't place component in selected area
		\begin{enumerate}
		\item[1.] System informs end-user that component can't be placed, because on that point already exists component.
		\item[2.] End of use case
		\end{enumerate}
}

\end{usecase}

\begin{usecase}

\addtitle{Use Case 2}{Changing flow output of pump} 

%Level: "user-goal" or "subfunction"
\addfield{Level:}{User-goal}

%Primary Actor: Calls on the system to deliver its services.
\addfield{Primary Actor:}{End-User}

%Preconditions: What must be true on start and worth telling the reader?
\addfield{Preconditions:}{User selected pump on flow network grid}

%Main Success Scenario: A typical, unconditional happy path scenario of success.
\addscenario{Main Success Scenario:}{
	\item System opens property sidebar
	\item User can change current flow and maximum flow of pump
	\item System updates values of component and calculates the flow through the network
	\item System updates the changes on the screen
}

%Extensions: Alternate scenarios of success or failure.
\addscenario{Extensions:}{
	\item[2.a] User gives a value for current flow larger than maximum flow
		\begin{enumerate}
		\item[1.] Current flow is set to the maximum flow
		\end{enumerate}
	\item[2.b] User gives a negative number or non-numeric value
		\begin{enumerate}
			\item[1.] Current flow is set to 0
		\end{enumerate}
}

\end{usecase}

\begin{usecase}

\addtitle{Use Case 3}{Changing flow output in splitter component} 

%Level: "user-goal" or "subfunction"
\addfield{Level:}{User-goal}

%Primary Actor: Calls on the system to deliver its services.
\addfield{Primary Actor:}{End-User}

%Preconditions: What must be true on start and worth telling the reader?
\addfield{Preconditions:}{User selected splitter on flow network grid}

%Main Success Scenario: A typical, unconditional happy path scenario of success.
\addscenario{Main Success Scenario:}{
	\item System opens property sidebar
	\item User set the division between outputs in the property sidebar using a slider
	\item System updates values of component and calculates the flow through the network
	\item System updates the changes on the screen
}

%Extensions: Alternate scenarios of success or failure.
%\addscenario{Extensions:}{

%}

\end{usecase}

\begin{usecase}

\addtitle{Use Case 4}{Creating connection between components} 

%Level: "user-goal" or "subfunction"
\addfield{Level:}{User-goal}

%Primary Actor: Calls on the system to deliver its services.
\addfield{Primary Actor:}{End-User}

%Preconditions: What must be true on start and worth telling the reader?
\addfield{Preconditions:}{Selected pipeline drawing tool from toolbar}

%Main Success Scenario: A typical, unconditional happy path scenario of success.
\addscenario{Main Success Scenario:}{
	\item User selects starting component
	\item User selects point on grid, through which pipeline will go
	\item User repeats step 2, until he has selected all points, through where pipeline will go
	\item User selects ending component
	\item System updates internal state
	\item System updates the changes on the screen
}

%Extensions: Alternate scenarios of success or failure.
\addscenario{Extensions:}{
	\item[1.a] Selected component already has no unused output
		\begin{enumerate}
		\item[1.] System informs end-user that the selected component can't be used.
		\item[2.] User returns to step 1.
		\item[3.] End of use case.
		\end{enumerate}
	\item[2.a] Selected point on grid is occupied by component.
		\begin{enumerate}
		\item[1.] System ignores the selected point.
		\end{enumerate}
}

\end{usecase}

\begin{usecase}

\addtitle{Use Case 5}{Deleting component on flow network grid} 

%Level: "user-goal" or "subfunction"
\addfield{Level:}{User-goal}

%Primary Actor: Calls on the system to deliver its services.
\addfield{Primary Actor:}{End-User}

%Preconditions: What must be true on start and worth telling the reader?
\addfield{Preconditions:}{Component selected}

%Main Success Scenario: A typical, unconditional happy path scenario of success.
\addscenario{Main Success Scenario:}{
	\item User right clicks on the component
	\item System shows context menu with options
	\item User clicks on Delete
	\item System removes selected component
	\item System removes pipelines connected to component, if component isn't pipeline
	\item System updates internal state
	\item System updates the changes on the screen
}

%Extensions: Alternate scenarios of success or failure.
%\addscenario{Extensions:}{

%}

\end{usecase}

\begin{usecase}
	
	\addtitle{Use Case 6}{Duplicating component on flow network grid} 
	
	%Level: "user-goal" or "subfunction"
	\addfield{Level:}{User-goal}
	
	%Primary Actor: Calls on the system to deliver its services.
	\addfield{Primary Actor:}{End-User}
	
	%Preconditions: What must be true on start and worth telling the reader?
	\addfield{Preconditions:}{Component selected}
	
	%Main Success Scenario: A typical, unconditional happy path scenario of success.
	\addscenario{Main Success Scenario:}{
		\item User right clicks on the component
		\item System shows context menu with options
		\item User clicks on Duplicate
		\item Go to Use case 1
		\item System sets the properties the same as original component
	}
	
	%Extensions: Alternate scenarios of success or failure.
	\addscenario{Extensions:}{
		\item[2.a] User right click on pipeline
		\begin{enumerate}
			\item[1.] Context menu don't show the option to duplicate
			\item[2.] End of use case.
		\end{enumerate}
	}
	
\end{usecase}

\begin{usecase}
	
	\addtitle{Use Case 7}{Moving component (exclusing pipeline) on flow network grid} 
	
	%Level: "user-goal" or "subfunction"
	\addfield{Level:}{User-goal}
	
	%Primary Actor: Calls on the system to deliver its services.
	\addfield{Primary Actor:}{End-User}
	
	%Preconditions: What must be true on start and worth telling the reader?
	\addfield{Preconditions:}{Component selected}
	
	%Main Success Scenario: A typical, unconditional happy path scenario of success.
	\addscenario{Main Success Scenario:}{
		\item User drags component to new position
		\item System redraws flow network
	}
	
	%Extensions: Alternate scenarios of success or failure.
	%\addscenario{Extensions:}{

	%}
	
\end{usecase}

\begin{usecase}

\addtitle{Use Case 8}{Load file} 

%Level: "user-goal" or "subfunction"
\addfield{Level:}{User-goal}

%Primary Actor: Calls on the system to deliver its services.
\addfield{Primary Actor:}{End-User}

%Preconditions: What must be true on start and worth telling the reader?
\addfield{Preconditions:}{User clicks on "Open" file button}

%Main Success Scenario: A typical, unconditional happy path scenario of success.
\addscenario{Main Success Scenario:}{
	\item System shows a file dialog-box.
	\item User selects file to open.
	\item System closes previous project.
	\item System loads the file.
}

%Extensions: Alternate scenarios of success or failure.
\addscenario{Extensions:}{
	\item[1.a] Another project is open
		\begin{enumerate}
			\item[1.] System asks if users wants, to save the project, if the user chooses cancel that will end the use case.
		\end{enumerate}
	\item[2.a] File can't be loaded.
		\begin{enumerate}
		\item[1.] System informs user that file can't be loaded and given the choice to stop or choose another file.
		\item[2.] End of use case.
		\end{enumerate}
}

\end{usecase}
\begin{usecase}

\addtitle{Use Case 9}{Save file} 

%Level: "user-goal" or "subfunction"
\addfield{Level:}{User-goal}

%Primary Actor: Calls on the system to deliver its services.
\addfield{Primary Actor:}{End-User}

%Preconditions: What must be true on start and worth telling the reader?
\addfield{Preconditions:}{User click the "Save" file button}

%Main Success Scenario: A typical, unconditional happy path scenario of success.
\addscenario{Main Success Scenario:}{
	\item System saves the file.
}
%Extensions: Alternate scenarios of success or failure.
\addscenario{Extensions:}{
	\item[1.a] Project file has not been saved on the device yet.
		\begin{enumerate}
		\item[1.] Go to Use Case 10
		\end{enumerate}
}
\end{usecase}
\begin{usecase}

\addtitle{Use Case 10}{Save file as a new file} 

%Level: "user-goal" or "subfunction"
\addfield{Level:}{User-goal}

%Primary Actor: Calls on the system to deliver its services.
\addfield{Primary Actor:}{End-User}

%Preconditions: What must be true on start and worth telling the reader?
\addfield{Preconditions:}{User clicks "Save as" button.}

%Main Success Scenario: A typical, unconditional happy path scenario of success.
\addscenario{Main Success Scenario:}{
	\item System shows a file dialog-box.
	\item User chooses the directory where the file will be stored, and names the file.
	\item System saves the file.
}
    
%Extensions: Alternate scenarios of success or failure.
\addscenario{Extensions:}{
	\item[1.a] User didn't intend to save file
		\begin{enumerate}
		\item[1.] User clicks "Cancel".
		\item[2.] End of use case
		\end{enumerate}
}
\end{usecase}